\documentclass[doc, biblatex]{apa6}

\usepackage[american]{babel}

\usepackage{csquotes}
\usepackage[style=apa6,sortcites=true,sorting=nyt,backend=biber]{biblatex}
\DeclareLanguageMapping{american}{american-apa}

\addbibresource{reference.bib}

\title{How Narrative Works in Outer Wilds}
\shorttitle{How Narrative Works in Outer Wilds}

\author{Zhixian He}
\affiliation{New York University}

\leftheader{He}

\abstract{This article briefly reviewed how curiosity works for animals and
  their relationship to exploration. Later we look at the definition of
  curiosity and exploration in the field of video games, and derive the
  definition of curiosity-driven exploration from that. We then analyze how
  exploration works in the game of progression and why many current open-world
  games are still feel linear and sequential. Then we have a closer look at how
  \textit{Legend of Zelda: Breath of the Wild} implement the curiosity-driven
  exploration, and its technique of ``telling stories of distant places''. Then
  we move our attention to the 2019 indie hit \textit{Outer Wilds}, to see how
  the game utilize all these ideas to build its own experience, and they are
  powering the narrative and integrate it into its gameplay.}

\begin{document}
\maketitle
\section{Introduction}
\textit{Outer Wilds} is a first-person action-adventure game developed by Mobius
Digital Games and published by Annapurna Interactive in May 2019. The game
features an unnamed alien astronaut as the player character exploring a small
solar system that is stuck in a 22-minute time loop. \textit{Outer Wilds} start
as Beachum's 2013 MFA thesis\parencite{beachum2013outer}, and the game won IGF
``Seumas McNally Grand Prize'' and ``Excellence in Design'' in
2015\parencite{igf2015}. They later spent 4 years to form a team and make the
game more polished.

\section{Curiosity and Exploration}
Berlyne said in his \textit{Curiosity and Exploration}, ``higher animals spend a
substantial portion of their time and energy on activities to which terms like
curiosity and play seem applicable''\parencite{berlyne1966curiosity}. Game
studies scholars are pretty familiar with the idea of play as an activity that
discharges energy\parencite{huizinga2020homo}, but this also applies to
curiosity.

There is a whole page on NASA's website, explaining why human beings would like
to explore. It says ``humans are driven to explore the unknown, discover new
worlds, push the boundaries of our scientific and technical limits, and then
push further''\parencite{nasa}. Curiosity is closely related to
exploration. Even though people usually explore for different kinds of reasons,
as Clint Hocking mentioned in his 2007 GDC interview by Ruberg, he said ``some
of them were motivated by money, some by patriotism or nationalism, some of them
by more of a pure desire to go where people hadn't
been''\parencite{ruberg2007clint}, but we would say that the strongest and the
purest motivation of exploration is curiosity.

To further discuss curiosity-driven exploration, we could first have a look at
Beachum's definition of some key terms\parencite{beachum2013outer}:
\begin{APAitemize}
\item \textbf{Curiosity} is defined as a need, thirst, or desire for knowledge.
\item \textbf{Exploration} refers to all activities concerned with gathering
  information about the environment.
\end{APAitemize}
Thus, we could explain curiosity-driven exploration as someone exploring the
environment with the primary objective of expanding their knowledge of the
surrounding world.

Even though a certain amount of video games happen in 3D spaces, many of them
never stress on the idea of exploration. For example, in \textit{Uncharted}
series, players are required to finish a linear series of tasks to beat the
game. The sequence of objectives are predefined, and players are not provided
with any choice. Juul defines the linear experience as progression
structure\parencite{juul2002open}. He points out that in progression structure,
designers have more control over the player experience, but on the other hand
this also limits players' freedom. Players are not playing actively. Although
they are actively making actions, but they are not thinking actively. They are
doing assigned tasks passively, and they are lack of curiosity.

The trend of the video game industry is to provide more freedom to players, and
we are seeing more and more video games labeled as ``open-world'' game. Indeed,
these games usually provide players a huge map to freely explore, but this
exploration is more like a marketing concept. Games like \textit{Grand Theft
  Auto} series and \textit{Assassin's Creed} series stress more on the mainline
story, which in another word, is still progression structure. Even though the
map itself is open for exploration, players are usually required to follow a
predefined sequence of actions and tasks to finish the game. In \textit{Witcher
  3}, players are promised that they could freely explore the whole map, but
usually they would find out that if they deviate too much from the desired
route, they may run into areas which have high level monsters that could easily
kill them.

What's more, the driving force of players' exploration in open-world games
usually is not the environment itself. AAA studios are investigating millions of
dollars on building believable 3D spaces but they do not serve any other
purpose.

\textit{Legend of Zelda: Breath of the Wild} is a positive example of
curiosity-driven exploration. In 2017, game director Fujibayashi explained how
they use terrain to guide players' exploration\parencite{cdedc2017}. He proposed
an idea of ``gravity'', which different places have different level of
attraction to players. These places themselves are meaningful, so that players
would like to explore. To some extent, is is similar to the idea of
curiosity-driven exploration. The game is not explicitly telling players where
to go, instead, they provide players with a huge amount of choices. Players are
able to see distant places, which are usually obvious is their sight. There
is a consensus between designers and players, that is these obvious distant
places could bring players something benificial.

\section{Telling Stories of Distant Places}
One interesting technique used in \textit{Outer Wilds} is by telling stories
of distant places. In \textit{Legend of Zelda: Breath of the Wild}, players
start the game with 12 photos of Hyrule on Link's tablet. One of the main quests
in this game is to find out all the places shot in these photos. Players could
check these photos at any time when they are on their journey. These photos
arouse players' curiosity. They give players some vague information about a
distant place. Players know the existence of the place and they know some
specific details about it, but they don't have any more concrete knowledge of
it. Link could retrieve memory pieces in these 12 spots, so there is some kind
of rewards for players who would like to explore.

In many modern open-world games, the overworld map is only used to navigate
players from place to place. Even though they provide players a huge world, they
are not encouraging players to explore it. Thus, telling stories of distant
places is a way to make players actively participate in the game, comparing to
cut-scene representation or doing repetitive quests.

\section{Narrative in Outer Wilds}
Although the long debate between ludology and narratology had been over for more
than 15 years, the game industry is still exploring the way to integrate the two
parts naturally. In 2009, Hocking pointed out the ludonarrative dissonance
problem in Bioshock, and this is still something a majority of video games are
suffering from\parencite{hocking2009ludonarrative}. Even in 2020, open-world
games like \textit{Ghost of Tsushima} is still using cinematic cut scenes to
tell stories. This doesn't mean it's not a proper way to implement the
narrative, but according to Frasca's definition, this is just using the
representation from cinematic media, not
simulation\parencite{frasca2003simulation}.

In \textit{Outer Wilds}, they push this idea further. The player character was
born in a village on a planet called Timber Hearth. After players finished the
tutorial part and can pilot the spaceship, they are allowed to freely explore
the whole solar system. The game provided players with several options at the
end of the tutorial part, but it never tells the player where they are required
to go, and what the goal of the game is. The major gameplay of the game is to
explore the universe, reading messages left by Nomai, another ancient alien
species, which died out many years ago. Players would face unknown natural
phenomenons and constructs built by Nomai.

\textit{Outer Wilds} has two kinds of places: one is referred to as
``Curiosities'', and the other is ``Points-of-Interest'', and this is how they
actually implement the idea of ``telling stories of distant places''. In the
game, ``Points-of-interest'' are places that are telling the existence of the
``Curiosities'' or other POIs. Usually, POIs provides enough information for
players to investigate on their own, so that players will never be stuck at any
point. There is always something waiting for them to explore. There are only 5
``Curiosities'' in the game, but each of them is holding the answer to a major
narrative question. All the POIs and Curiosities together consist of a huge
``Web of Curiosities''. A interesting thing is, these Curiosities are not
guarded with a mechanic lock, which means that players have to complete some
prerequisite task to have access to these places. They are only guarded with
players' knowledge of the solar system. Usually, players wouldn't be able to
enter these Curiosities if they haven't gained related knowledge.

\begin{figure*}
  \centering
  \fitfigure{shiplog}
  \caption{Outer Wilds full shiplog}
\end{figure*}

By using this ``Web of Curiosities'', \textit{Outer Wilds} naturally embed all
the narrative into its gameplay. The stories are the rewards of players'
exploration, and these stories are pointing to other places that are waiting for
players to explore. Traditional embedded narrative in video games serves as a
supplement to the background of the game world. For example, in \textit{Fallout
  4}, players could collect letters and tapes to have a better knowledge of what
had happened in this world, but they are not the key element to the main
story. However, the way \textit{Outer Wilds} uses this narrative structure, is
just how Jenkins talked about detective stories in Game Design as Narrative
Structure. In this essay, Jenkins argues that detective stories use embedded
narrative because ``they motivate the player's active examination of clues and
exploration of spaces and provide a rationale for our efforts to reconstruct the
narrative of past events''\parencite{jenkins2004game}. In \textit{Outer Wilds},
players are not experiencing a predefined story. Instead, they main goal is to
figure out what has happened. Although this may break the structure aspect of
the narrative design, which make designers have weaker control over the
narrative structure, this ``Web of Curiosities'' indeed organized everything
into a whole.

From micro aspect, the sequence of the knowledge doesn't matter anymore. Players
could choose where they want to explore first, and where to go later. But from
macro aspect, the sequence is guaranteed by the ``Web of Curiosities'', since in
the web, the knowledge players learnt from POIs are always pointing to
``Curiosities''. Players will gradually learn where the game ends while they
explore the solar system. As we mentioned before, each of the ``Curiosities'' is
a major part of the narrative. After players get a full knowledge of the history
of Nomai and the solar system, they would know where to go.

To conclude, the knowledge of the universe is players' real ability. The
whole process is:
\begin{APAitemize}
\item Players explore the solar system and they encounter some POIs.
\item POIs tell the existence of some Curiosities; players travel to several
  POIs to gain full knowledge of the access method to a specific Curiosity.
\item Players visit the curiosity and gain new knowledge of the narrative part,
  either the history of Nomai or the solar system or an explanation to a natural
  phenomenon.
\item The newly gained story is also a part of players knowledge of the solar system,
which could guide their further exploration as well.
\end{APAitemize}
The Web of Curiosities makes sure players won't be stuck or get lost as each of
these elements are connected, which means they are always pointing to another
one.

This curiosity-driven exploration integrates the narrative into the gameplay
itself, and the narrative also helps players to explore this virtual space
on the level of mechanics. Unlike cinematic storytelling, the narrative method in
\textit{Outer Wilds} never breaks the magic circle. All the narrative in this
game are either conversation with other astronauts or reading written messages
left by Nomai. Both of them happen in the fictional world and are not pushing
you away from the controller like the cut scenes do.

\section{Conclusion}
The narrative in \textit{Outer Wilds} could only work in video games. Juul once
argues in Games Telling Stories that ``There is an inherent conflict between the
\textit{now} of the interaction and the \textit{past} or 'prior' of the
narrative''\parencite{juul2001games}. But exploring the story of what has
happened to the virtual space skillfully solved this problem. Games like
\textit{Return of the Obra Dinn} also uses embedded narrative to tell a story
which had already happened. Video games nowadays are trying to focus more on
emergent narrative, as this could provide endless stories which could make
player experience more diverse, but \textit{Outer Wilds}'s curiosity-driven
exploration designed around the Web of Curiosities successfully integrate the
embedded narrative into the gameplay, making the narrative and the mechanics
feel more like a whole. This kind of ludonarrative resonance is like the holy
grail of game design, and there are still so many spaces for us to explore.
\printbibliography
\end{document}
